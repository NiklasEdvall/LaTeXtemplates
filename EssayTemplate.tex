\documentclass[12pt]{article} %document class and font size
\usepackage[utf8]{inputenc} %character encoding
\usepackage{fancyhdr} %Package for header and footer
\usepackage{graphicx} %images
\usepackage{caption} %format image captions (e.g. column width)

%Paper size and margins, guide: https://www.overleaf.com/learn/latex/page_size_and_margins
\usepackage[a4paper,width=150mm,top=30mm,bottom=25mm]{geometry} 

%Package for bibliography, replace example_library.bib with own bibliography (e.g. export as biblatex from Zotero)
\usepackage[backend=biber,style=authoryear,sorting=nyt]{biblatex} %Or style=numeric, sorting=none
\addbibresource{example_library.bib}

%Image folder
\graphicspath{ {./images/} }

%Specify header and footer
\pagestyle{fancy}
\fancyhf{}
\rhead{\today} %Prints current date
\lhead{Firstname Lastname}
\renewcommand{\headrulewidth}{0pt} % Zero width header separator line
\rfoot{} % No foot (i.e page number)

\newcommand\tab[1][1.5cm]{\hspace*{#1}} %define tab size

\setlength{\headsep}{15mm} % Space from header to title





\begin{document}

\begin{center}
\LARGE{The title of this document goes here}
\end{center}

\vspace{5mm} % Distance, title to first section

\section*{The first section} %Star (*) blocks section title from being numbered.
The first section can contain some text. If one sentence is not enough, it can always be followed up by a second one.\\

\section*{Then the second}
To follow up on the very short first section, some value may exist in providing a second section. Especially in a template document like this one, repetition can be what really makes the point. \\

Then, all of a sudden, something new (Figure 1). However, I still haven't really figured out how to reference figures internally yet.\\

The second section may even contain a second paragraph, mainly to remind the writer on the use of $\backslash\backslash$. However, a more common way to escape special characters (i.e. \$ or \#) is to use a single $\backslash$. \\

\begin{figure}[h]
\captionsetup{labelfont=bf, width=100mm, font=footnotesize}
\includegraphics[width=100mm]{images/picture}
\centering
\caption{Here goes the text that explains what your'e looking at. In this case a work of art.}
\end{figure} \\

\section*{Final section}
Finally then, we get to the point where we can put in a small code example, as below. And really, this template should be updated asap to also include a figure or two. Anyway, here is some intended text in a code-looking font:\\

{\fontfamily{qcr}\selectfont
sub-123/ \\
\tab ses-210303/ \\
\tab \tab meg/ \\
\tab \tab \verb|sub-123_ses-210303_task-sensmot_run-1_meg.fif| \\
\tab \tab \verb|sub-123_ses-210303_task-sensmot_run-1_meg.json| 
}\\

To wrap up then, what about citations and a bibliography? Speaking about figures, as mentioned above, maybe the template figure could be a Bland-Altman plot, first described now almost 30 years ago (\cite{altman_measurement_1983}).

\break %New page

\printbibliography %insert bibliography

\end{document}
